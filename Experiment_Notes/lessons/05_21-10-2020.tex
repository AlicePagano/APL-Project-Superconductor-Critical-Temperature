\documentclass[../main/main.tex]{subfiles}

\newdate{date}{21}{10}{2020}


\begin{document}

\marginpar{ \textbf{Laboratory 5.} \\  \displaydate{date}. \\ Compiled:  \today.}


\section{Raffreddamento circuito con azoto liquido}

Oggi abbiamo iniziato a preparare il cablaggio per il criostato. Per fare ciò abbiamo saldato i vari fili per assemblare il connettore.

Prima di mettere il superconduttore nel criostato, dobbiamo essere sicuri della corretteza del circuito. Per fare ciò mettiamo il superconduttore in una cassetta del gelato e lo ricopriamo di azoto liquido, questo si raffredda e dovremmo essere in grado di vedere il salto di potenziale (il punto critico del superconduttore (?)). Quindi abbiamo preparato i connettori apposta per collegare il superconduttore al circuito sulla basetta.


Per il criostato, dobbiamo accoppiare termicamente dito freddo e il nostro campione. Abbiamo due termometri uno del criostato e l'altro della temperatura del campione. Se scendiamo troppo giù di temperatura l'accoppiamento rischia di non essere ottimale e vediamo temperature diverse.




\end{document}
