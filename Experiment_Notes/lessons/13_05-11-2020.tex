\documentclass[../main/main.tex]{subfiles}

\newdate{date}{05}{11}{2020}


\begin{document}


\chapter{Fase di misura}

\section{Test preliminare}

\marginpar{ \textbf{Laboratory 13.} \\  \displaydate{date}. \\ Compiled:  \today.}

Dopo che siamo riuscite a risolvere il problema del ponte (termometro), abbiamo iniziato a testare il circuito. Abbiamo acceso il criostato. Nella prima prova di raffreddamento abbiamo notato che la temperatura minima ottenuta nella camera (misurata con il Pt100 del dito freddo) risulta essere:
\begin{equation*}
  T_{min}^{Pt100} = 63 \, \text{K}
\end{equation*}
Mentre la temperatura ottenuta dalla calibrazione del ponte:
\begin{equation*}
  T_{min}^{ponte} = 120 \, \text{K}
\end{equation*}
Per misurare quest'ultima abbiamo calibrato il ponte e misurato la resistenza dell'helipot. In queesto caso l'helipot misurava 37 giri, che corrispondono guardando la tabella a circa quella temperatura.
Invece vediamo che la tensione misurata risulta essere:
\begin{equation*}
  V_{sc} = 80 \, \text{mV}
\end{equation*}
L'acquisizione è stata salvata come "prova2.csv", in particolare in questo dataset fino a tempo 3200 abbiamo sbagliato, in quanto non avevamo acceso il generatore per dare segnale all'amplificatore.




\end{document}
