\documentclass[../main/main.tex]{subfiles}

\newdate{date}{12}{11}{2020}


\begin{document}

\marginpar{ \textbf{Laboratory 17.} \\  \displaydate{date}. \\ Compiled:  \today.}

\subsection{Potenziale \( \pmb{V_0 = 6} \) V}

In questa lezione abbiamo fissato:
\begin{itemize}
\item \( V_0 = 6 \) V;
\item \( R_{fili} = 6.5 \);
\item una pressione approssimativamente circa \( p=1.2\times10^{-4} \) mbar;
\item un guadagno del lockin di \( \times 2 \);
\item un potenziale del ponte di \( V_{ponte}=5.65 \) V (da misurare! infatti questo è solo il valore riportato);
\item il fondo scala del multimetro è di \( 6 \) mV.
\end{itemize}

Abbiamo effettuato le misure nel seguente modo:
\begin{itemize}
\item abbiamo raffreddato il campione da una temperatura \( T_c= 251.052\) K, superando la temperatura critica (osservando il salto sempre a 110 K) fino ad una temperatura del dito freddo di 53.55 K;
\item abbiamo riscaldato da una temperatura del dito freddo di 76.71 K fino a 143.9 K;
\item abbiamo spento il potenziale in ingresso e raffreddato da una temperatura di 129.425 K fino a 56.73 K (misura dell'offset di raffreddamento);
\item sempre con il potenziale in ingresso spento abbiamo ririscaldato da una temperatura di 88.882 K fino a 130.583 K (misura dell'offset di riscaldamento).
\end{itemize}



\end{document}
